\documentclass[letterpaper,10pt,oneside]{article}

\usepackage{hyperref}

\title{Staffing Narative}
\author{Jackson Wills}

\begin{document}

\maketitle

\tableofcontents

\clearpage

\section{Before The Broadcast}
A lot of preparation goes into news casts. This is what everyone is doing before the show even starts.

\subsection{Executive Producer}
The executive producer needs to first staff the show. There must be competent people at everyposition. This is a team sport. They also need to decide what stories are going to air. They need to gauge what stories are important to their viewers. They select the guests that will come onto the show, either to be interviewed or to be a member of a panel. These guests need to be contacted and the logists of making sure that they can be present during the show. They will work with the anchors to select the questions that will be asked to guests. They make sure that all packages are done and look good before they go on air. They set deadlines and make sure they are met.
The executive producer must make sure that graphics have completed the graphics package for the show. They need to make sure the show is going to be 30 minutes long. A backup store or two may be necessary if something goes wrong.

\subsection{Director}
The director needs to know the show rundown very well. They are backtiming the whole show in their head the whole time. They need to make sure that all the breaks are taken and the show is exactly 30 minutes long.
The director must also be in close contact with the technical director so they can work out any possible difficult parts of the show. The director must make sure that graphics have completed the graphics package for the show.
They'll also make sure that all packages are done and look good before they go on air. They need to know when which lower thirds need to be inserted and need to have talked to the graphics person about this.

\subsection{Technical Director}
The technical director must have a very good understanding of the show rundown. They need to know where all the packages are located, and work with graphics on packages and lower thirds. The technical director must also be in close contact with the director so they can work out any possible difficult parts of the show. The technical director needs to know which inputs are which, and if there is audio associated with any of them. They need to talk to audio about the packages.

\subsection{Graphics}
A graphics package must be produced that is clear, aesthetically pleasing, and easy to read. Lower thirds must be prepared for everyone specified on the script (but it is basically everyone in the show and in the packages).

\subsection{Teleprompter}
The teleprompter needs to write the script. The script must be clear and easy to understand when read aloud. If hard to pronounce words are included, phonetics must be given. The teleprompter will develop a relationship with the anchors as to the cadence and voice type of the anchors.

\subsection{Audio}

\subsection{Anchors}
The anchors need to look over the script and become familiar with the stories they are going to present. They will work with the executive producer to select the interview questions that will be asked to guests.

\subsection{Floor Director}

\subsection{Camera Operators}
The camera operators need to know the

All shots need to be white balanced. This is especially important for on site shooting.


%%%%%%%%%%%%%%%%%%%%%%%%%%%%%%%%


\section{During The Broadcast}
A lot is going on during a broadcast. Here is what everyone is doing during the show.
\subsection{Executive Producer}

\subsection{Director}
The director needs to be consistent and clear with the nomenclature used to comunicate with the rest of the crew.

\subsection{Technical Director}
The technical director must have a very good understanding of the show rundown. They need to know where all the packages are located, and work with graphics on packages and lower thirds. The technical director must also be in close contact with the director so they can work out any possible difficult parts of the show.

\subsection{Graphics}
A graphics package must be produced that is clear, aesthetically pleasing, and easy to read. Lower thirds must be prepared for everyone specified on the script (but it is basically everyone in the show and in the packages).

\subsection{Teleprompter}
The teleprompter needs to write the script. The script must be clear and easy to understand when read aloud. If hard to pronounce words are included, phonetics must be given. The teleprompter will develop a relationship with the anchors as to the cadence and voice type of the anchors.

\subsection{Audio}

\subsection{Anchors}
The anchors need to look over the script and become familiar with the stories they are going to present.

\subsection{Floor Director}

\subsection{Camera Operators}





\end{document}
