\documentclass[letterpaper,10pt,oneside]{article}

\usepackage{hyperref}

\title{Staffing Narative}
\author{Jackson Wills}

\begin{document}

\maketitle

\tableofcontents

\clearpage

\section{Before The Broadcast}
A lot of preparation goes into news casts. This is what everyone is doing before the show even starts.

\subsection{Executive Producer}
The executive producer needs to first staff the show. There must be competent people at every position. This is a team sport. They also need to decide what stories are going to air. They need to gauge what stories are important to their viewers. They select the guests that will come onto the show, either to be interviewed or to be a member of a panel. These guests need to be contacted and the logists of making sure that they can be present during the show need to be figured out. They will work with the anchors to select the questions that will be asked to guests. They make sure that all packages are done and look good before they go on air. They set deadlines and make sure they are met.
The executive producer must make sure that graphics have completed the graphics package for the show. They need to make sure the show is going to be exactly 30 minutes long. A backup story or two may be necessary if something goes wrong.

\subsection{Director}
The director needs to know the show rundown very well.
The director must also be in close contact with the technical director so they can work out any possible difficult parts of the show. The director must make sure that graphics have completed the graphics package for the show.
They'll also make sure that all packages are done and look good before they go on air. They need to know when which lower thirds need to be inserted and need to have talked to the graphics person about this. The director needs to have a plan for what shots are going to be taken and when. They need to know where all the packages and info graphics are.

\subsection{Technical Director}
The technical director must have a very good understanding of the show rundown. They need to know where all the packages are located, and work with graphics on packages and lower thirds. They need to talk to graphics and the director to make sure that everything that needs to be there is there, and that everyone knows where those things are. The technical director must also be in close contact with the director so they can work out any possible difficult parts of the show. The technical director needs to know which inputs are which, and if there is audio associated with any of them. They need to talk to audio about the packages.

\subsection{Graphics}
A graphics package must be produced that is clear, aesthetically pleasing, and easy to read. Lower thirds must be prepared for everyone specified on the script (but it is basically everyone in the show and in the packages). Audio and TD needs to know where the graphics person is going to put the packages. They need to know who is speaking, and when, and have lower thirds for them. They also need to make any info graphics.

\subsection{Teleprompter}
The teleprompter needs to write the script. The script must be clear and easy to understand when read aloud. If hard to pronounce words are included, phonetics must be given. The teleprompter will develop a relationship with the anchors as to the cadence and voice type of the anchors. The teleprompter needs to have worked with the anchors so that the teleprompter knows how fast to move the prompter during the show.

\subsection{Audio Technician}
The audio technician needs to work with the audio crew to set up all the audio cables. Every input must be known to the audio technician. They must also be familiar with the rundown so that they know when each input will be used. Microphone checks must be be done prior to air. The audio inputs for each block should be known and marked on the audio board.

\subsection{Anchors}
The anchors need to look over the script and become familiar with the stories they are going to present. They will have marked up the script for emphasis and emotion. They will work with the executive producer to select the interview questions that will be asked to guests. The anchors need to know roughly what thay are going to say during non scripted portions. They anchors should communicate to each other about what they are going to say during these unscripted portions so that nothing is repeated.

\subsection{Floor Director}
The floor director is the liason between the director and the camera operators. They need to know what shots the director is looking for. They need to know the rundown and what shots the director wants during all the blocks. They need to make sure everything on set is the way it needs to be for the show. They need to make sure all the camera shots are well composed - they will work the executive producer on composition.

\subsection{Camera Operators}
The camera operators need to know the show rundown. They need to have their opening shot ready before we go on air. They also need to know whether or not they will need to make any movements on air. They will have talked to the director about the shots they are going to get. All shots need to be white balanced. This is especially important for on shots that are not in the studio.


%%%%%%%%%%%%%%%%%%%%%%%%%%%%%%%%
\vspace{25 mm}

\section{During The Broadcast}
A lot is going on during a broadcast. But a lot of the real work was done before the show even started. Here is what everyone is doing during the show. They are watching to show for style and composition.
\subsection{Executive Producer}
They are backtiming the whole show in their head the whole time. They need to make sure that all the breaks are taken and the show is exactly 30 minutes long.

\subsection{Director}
The director needs to be consistent and clear with the nomenclature used to comunicate with the rest of the crew. They need to make sure that every shot they put on the air looks good before they put it on air. They are backtiming the whole show in their head the whole time. They need to make sure that all the breaks are taken and the show is exactly 30 minutes long. If something goes wrong, they are the one who needs to come up with a plan and order the crew efficiently and effectively.

\subsection{Technical Director}
They must be very in tune with the directors orders. They already know where everything is. They have worked with audio and graphics before the show, and this work should make show time very smooth.

\subsection{Graphics}
Graphics must be on their toes. They need to be paying attention to what is happening with the show and the director and technical director so that they know when to change the inputs. They need to know what is going to be used next by the director and put that where it needs to be so that the technical director can use it.

\subsection{Teleprompter}
The teleprompter needs to be moving the prompter at the proper rate. They need to be in tune with any changes in the show, and they need to respond accordingly.

\subsection{Audio Technician}
During the broadcast there may be a need for audio cables to be changed. New inputs, or a different person on a microphone would mean that these cables need to be rewired. The A2 needs to do this without ruining any shots, or messing with the camera wires. If this happens, microphone checks would need be done again. This is tricky during the show, but it can be done as long as the audio technician is sure to output the sound only in audition and \textbf{not} in program. hello this shouldnt be bold...
\subsection{Anchors}
The anchors need to present to stories from the teleprompter and adlib the unscripted portions. Interviews and panels should be presented professionally. Preparation before the show is very important for the anchors. During the show, this preparation should make the actual show time easy.

\subsection{Floor Director}
The floor director needs to make sure everything that was supposed to happen in the studio actually happens once we are on air. They are commanding people to do what they need to do when they need to do it. This includes zooms, pans, and moving intire cameras to get new shots.

\subsection{Camera Operators}
The camera operators need to get the shots that were predecided by the executive producer, director, and floor director. They need to be in tune with commands from the director about the composition of the shots.





\end{document}
